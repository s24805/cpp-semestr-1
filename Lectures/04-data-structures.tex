\documentclass[aspectratio=169,10pt]{beamer}

\usepackage[utf8]{inputenc}
\usepackage{textcomp}
\usepackage[polish]{babel}
\usepackage{amsthm}
\usepackage{graphicx}
\usepackage[T1]{fontenc}
\usepackage{scrextend}
\usepackage{hyperref}
\usepackage{xcolor}
\usepackage{geometry}
\usepackage{listings}
\usepackage{epigraph}

\renewcommand{\epigraphsize}{\scriptsize}

\usetheme{-bjeldbak/beamerthemebjeldbak}

\definecolor{xbarcolor}{HTML}{000000}
\setbeamercolor{lower separation line head}{bg=xbarcolor}
\setbeamercolor{lower separation line foot}{bg=xbarcolor}

\title{Podstawy programownia (w języku C++)}
\subtitle{Struktury danych}
\author{Marek Marecki}
\institute{Polsko-Japońska Akademia Technik Komputerowych}

\lstset{basicstyle=\ttfamily\color{black},
columns=fixed,
escapeinside={[*}{*]},
inputencoding=utf8,
extendedchars=true,
moredelim=**[is][\color{red}]{@}{@},
moredelim=**[is][\color{gray}]{`}{`},
moredelim=**[is][\color{olive}]{$}{$}}

\begin{document}

{%
    \setbeamertemplate{headline}{}
    \frame{\titlepage}
}

\section{Struktury danych}

\begin{frame}
    \frametitle{Po co?}
    \framesubtitle{Struktury danych}

    Struktury (typy) danych wykorzystuje się jeśli zachodzi potrzeba zgrupowania
    pewnych danych mających \emph{wspólny raison
    d'être\footnote{\url{https://en.wiktionary.org/wiki/raison_d'être}}};
    reprezentujących zestaw \emph{danych} i \emph{operacji} na tych danych,
    których przechowywanie osobno nie miałoby większego sensu.

    \vspace{1em}

    Biblioteka standardowa zawiera strukturę danych {\tt std::vector}, która
    reprezentuje tablicę o zmiennej długości. Grupuje ona \emph{dane} (zawartość
    i rozmiar), oraz \emph{operacje} (np. dodawanie, usuwanie i dostęp do
    elementów).
\end{frame}

\begin{frame}[fragile]
    \frametitle{Składniki}
    \framesubtitle{Struktury danych}

    Struktury (typy) danych składają się przede wszystkim z \textbf{pól} (ang.
    \emph{fields} lub \emph{member variables}), czasem zwanych zmiennymi lub
    stałymi składowymi, i \textbf{funkcji składowych} (ang. \emph{member
    functions}).

    {\scriptsize
    \begin{lstlisting}
    struct Cat {
        size_t number_of_lives { 1 };     $// member variable (field)$
        bool const brings_luck { false }; $// member constant (field)$

        auto make_sound() const -> void;  $// member function$
    };

    auto Cat::make_sound() const -> void
    {
        std::cout << "Meow! I have "
            << number_of_lives << " lives left.\n";
    }
    \end{lstlisting}}
\end{frame}

\begin{frame}[fragile]
    \frametitle{Pola}
    \framesubtitle{Struktury danych}

    Pola służą do przechowywania \emph{danych}, które dany typ grupuje. Na
    przykład rozmiar i zawartość wektora.

    Pola mogą być zmienne - jeśli ich wartości mogą podlegać modyfikacji w
    trakcie życia obiektu danego typu (np. rozmiar {\tt std::vector}), albo
    stałe - jeśli nie podlegają modyfikacji (np. rozmiar {\tt std::array}).

    \vspace{1em}

    Definiując pole trzeba użyć starego stylu deklaracji, czyli poprzedzić nazwę
    typem, a nie słowem kluczowym {\tt auto}:

    {\scriptsize
    \begin{lstlisting}
    size_t number_of_lives { 1 };
    \end{lstlisting}}
\end{frame}

\begin{frame}
    \frametitle{Funkcje składowe}
    \framesubtitle{Struktury danych}

    Funkcje składowe służą do wykonywania \emph{operacji} na danych zgrupowanych
    przez dany typ (np. {\tt std::vector::push\_back}), lub do ogólnej
    interakcji z nim (np. {\tt Cat::make\_sound}).
\end{frame}

\begin{frame}[fragile]
    \frametitle{{\tt class} vs {\tt struct}}
    \framesubtitle{Struktury danych}

    Jedyna różnica to domyślna \emph{widoczność} pól i funkcji składowych. Pola
    klas ({\tt class}) są prywatne, a struktur ({\tt struct}) publiczne.
    Widoczność można zmieniać słowami kluczowymi {\tt private} i {\tt public}.

    {\scriptsize
    \begin{lstlisting}
    class Foo {
        @public@:
        bool now_you_see_me { true };

        @private@:
        bool now_you_dont { false };
    };
    \end{lstlisting}}
\end{frame}

\begin{frame}[fragile]
    \frametitle{Widoczność pól}
    \framesubtitle{Struktury danych}

    Po co jest widoczność? Niektóre pola w strukturze mogą nie być przeznaczone
    dla ''użytkowników'' lub wymagać zachowania pewnych warunków. Jeśli jest
    taka potrzeba zawsze można udzielić dostępu do pola przez funkcje składowe.

    {\tiny
    \begin{lstlisting}
    class Hour {
        unsigned value { 0 };  $// private by default$
        public:
        auto what_time_is_it() const -> unsigned;
        auto increase_time() -> void;
    };
    auto Hour::what_time_is_it() const -> unsigned
    {
        reurn value;
    }
    auto Hour::increase_time() -> void
    {
        ++value;
        if (value > 23) {  $// make sure the hour wraps after 23$
            value = 0;
        }
    }
    \end{lstlisting}}
\end{frame}

\begin{frame}[fragile]
    \frametitle{Konstruktor}
    \framesubtitle{Struktury danych}

    Konstruktor (ang. \textit{constructor}) jest specjalną funkcją składową
    odpowiedzialną za ''przygotowanie'' struktury danych do użycia. Konstruktor
    może zasygnalizować niemożność utworzenia instancji struktury (np. z powodu
    niewłaściwych wartości argumentów) używając wyjątków\footnote{ponieważ
    konstruktor jako takie nie zwraca wartości więc nie ma jak inaczej
    zasygnalizować błędu}.

    {\tiny
    \begin{lstlisting}
    class Hour {
        $// code from last slide here...$
        public:
        Hour(unsigned);
    };
    Hour::Hour(unsigned @v@)
        : value{v} $// initialise the field using values from parameters$
    {
        @if@ (@v > 23@) { $// throw if they don't make sense$
            @throw@ std::out_of_range{"hour value cannot exceed 23"};
        }
    }
    \end{lstlisting}}
\end{frame}

% FIXME use this when talking about resources, I/O, and threads
% \subsection{RAII}

% \begin{frame}
%     \frametitle{RAII - Resource Acquisition Is Initialisation}
%     \framesubtitle{Struktury danych}

%     (ang. \textit{destructor})
% \end{frame}

% \begin{frame}
%     \frametitle{Destruktor}
%     \framesubtitle{Struktury danych}

%     (ang. \textit{destructor})
% \end{frame}

% \begin{frame}
%     \frametitle{Kopiowanie}
%     \framesubtitle{Struktury danych}

%     (ang. \textit{destructor})
% \end{frame}

% \begin{frame}
%     \frametitle{Przenoszenie}
%     \framesubtitle{Struktury danych}

%     (ang. \textit{destructor})
% \end{frame}

% \begin{frame}
%     \frametitle{Rule of five}
%     \framesubtitle{Struktury danych}

%     (ang. \textit{destructor})
% \end{frame}

% \begin{frame}
%     \frametitle{Rule of zero}
%     \framesubtitle{Struktury danych}

%     (ang. \textit{destructor})
% \end{frame}

\section{Pola}

\begin{frame}[fragile]
    \frametitle{Zmienne}
    \framesubtitle{Pola}

    Definicja zmiennego pola wygląda tak jak definicja każdej innej zmiennej.
    Załóżmy, że chcemy stworzyć strukturę opisującą kota. Jak wiadomo, koty mają
    9 żyć do wykorzystania:

    {\small
    \begin{lstlisting}
    struct Cat {
        constexpr static unsigned MAX_LIVES = 9;
        unsigned lives_left { MAX_LIVES };
    };
    \end{lstlisting}}
\end{frame}

\begin{frame}
    \frametitle{Inicjalizacja zmiennych składowych}
    \framesubtitle{Pola}

    Zmienne składowe można zainicjalizować na kilka sposobów:
    \begin{enumerate}
        \item przypisując im wartość w liście inicjalizującej składowe (ang.
            \emph{member initialiser list})
        \item przypisując im wartość w ciele konstruktora
        \item pozostawiając ich wartości domyślne zdefiniowane w ciele struktury
    \end{enumerate}
\end{frame}

\begin{frame}[fragile]
    \frametitle{Inicjalizacja zmiennych składowych (c.d.)}
    \framesubtitle{Pola}

    {\scriptsize
    \begin{lstlisting}
    struct Cat {
        constexpr static unsigned MAX_LIVES = 9;
        unsigned lives_left { MAX_LIVES }; $// default value provided$

        Cat() = default; $// default ctor$
        Cat(unsigned const);
    };
    Cat::Cat(unsigned const ll)
        : lives_left{ll} $// use member initialiser list$
    {}
    \end{lstlisting}}

    Jeśli definiujemy konstruktor to kompilator nie wygeneruje konstruktora
    domyślnego (z pustą listą parametrów). Jeśli chcemy w takiej sytuacji
    skorzystać z domyślnego konstruktora to możemy albo zdefiniować go
    samodzielnie, albo użyć konstrukcji {\tt = default}.
\end{frame}

\begin{frame}[fragile]
    \frametitle{Dostęp}
    \framesubtitle{Pola}

    Aby dostać się do pola struktury należy użyć operatora `{\tt .}' (kropki):

    {\small
    \begin{lstlisting}
    auto a_cat = Cat{}; $// an ordinary cat$

    std::cout << a_cat@.@lives_left << "\n";
    \end{lstlisting}}

    Jeśli pole jest zmienne, można je zmodyfikować:

    {\small
    \begin{lstlisting}
    a_cat@.@lives_left @-=@ 1; $// the cat died$
    \end{lstlisting}}

    lub przypisać mu całkiem nową wartość:

    {\small
    \begin{lstlisting}
    a_cat@.@lives_left @=@ 666; $// cat from hell$
    \end{lstlisting}}
\end{frame}

\begin{frame}[fragile]
    \frametitle{Stałe}
    \framesubtitle{Pola stałe (stałe składowe)}

    Definicja stałego pola wygląda tak jak definicja każdej innej stałej.
    Kontynuując przykład z kotem:

    {\scriptsize
    \begin{lstlisting}
    struct Cat {
        constexpr static unsigned MAX_LIVES = 9;
        unsigned lives_left { MAX_LIVES };

        std::string @const@ name;

        Cat() = default;
        Cat(unsigned const);
    };
    \end{lstlisting}}

    Wartości pół stałych nie można\footnote{Oh,
    rly?\label{hackerman_const_field_mutation}} zmienić, nawet w ciele
    konstruktora. Do ich inicjalizacji służy specjalna notacja.
\end{frame}

\begin{frame}[fragile]
    \frametitle{Inicjalizacja stałych składowych}
    \framesubtitle{Pola stałe (stałe składowe)}

    Żeby mieć możliwość inicjalizacji stałego pola różnymi wartościami w różnych
    instancjach struktury potrzebny jest konstruktor przyjmujący jako parametr
    wartość, którą stałe powinno być zainicjalizowane:

    {\scriptsize
    \begin{lstlisting}
    struct Cat {
        constexpr static unsigned MAX_LIVES = 9;
        unsigned lives_left { MAX_LIVES };

        @std::string const name@;

        Cat() = default;
        Cat(unsigned const, @std::string@);
    };
    Cat::Cat(unsigned const ll, @std::string n@)
        : lives_left{ll}
        , @name{std::move(n)}@
    {}
    \end{lstlisting}}
\end{frame}

\begin{frame}[fragile]
    \frametitle{Dostęp}
    \framesubtitle{Pola stałe (stałe składowe)}

    Do stałych pól struktury można się dostać za pomocą operatora `{\tt .}'
    (kropki):

    {\small
    \begin{lstlisting}
    auto mr_snuggles = Cat{ 9, "Mr. Snuggles" };

    std::cout << mr_snuggles.name << "\n";
    \end{lstlisting}}

    Możliwy jest jednak dostęp wyłącznie do odczytu, a kompilator zapobiegnie
    ich przypadkowej modyfikacji:

    {\small
    \begin{lstlisting}
    mr_snuggles.name = "Evil Elvis"; @error!@
    \end{lstlisting}}
\end{frame}

\begin{frame}
    \frametitle{Dostęp - modyfikacja stałych składowych?!}
    \framesubtitle{Pola stałe (stałe składowe)}

    \begin{figure}[!htp]
        \centering
        \includegraphics[width=9cm]{hackerman}
        \caption{Hackerman\footnote{Kung Fury (2015), \url{https://www.imdb.com/title/tt3472226/}}}
    \end{figure}
\end{frame}

\begin{frame}[fragile]
    \frametitle{Dostęp - modyfikacja stałych składowych?!}
    \framesubtitle{Pola stałe (stałe składowe)}

    Jeśli chce się zaimponować cioci w odwiedzinach, albo babci na święta to
    można pokazać im jak \textsc{oszukać kompilator} i zmodyfikować
    \textsc{stałe składowe}!

    {\scriptsize
    \begin{lstlisting}
    `std::cout << mr_snuggles.name << "\n";`

    auto wait_its_illegal = &mr_snuggles.name; $// pointer to member$
    @*const_cast<std::string*>(wait_its_illegal)@ = "Evil Elvis";

    `std::cout << mr_snuggles.name << "\n";`
    \end{lstlisting}}

    {\tiny Przy okazji widać też jak można pobrać \emph{wskaźnik do składowej}
    (ang. \emph{pointer to member}). Nie jest to coś co często się przydaje, ale
    na pewno częściej niż modyfikacja stałych składowych.}
\end{frame}

\section{Funkcje}

\begin{frame}
    \frametitle{Definiowanie funkcji składowych}
    \framesubtitle{Funkcje składowe}

    Funkcję należy zadeklarować wewnątrz struktury, a następnie zdefiniować poza
    strukturą.

    \vspace{1em}

    Definicje struktur (ich nazwy, pola, funkcje składowe) znajdują się
    zazwyczaj w plikach nagłówkowych, np. {\tt Cat.h} (wyjątek to statyczne
    zmienne składowe, które trzeba dodatkowo zadeklarować w pliku z
    implementacją).\newline
    Definicje funkcji składowych znajdują się zazwyczaj w plikach z ich
    implementacją, np. {\tt Cat.cpp} (wyjątek to np. funkcje {\tt inline} albo
    szablony).
\end{frame}

\begin{frame}[fragile]
    \frametitle{Definiowanie funkcji składowych (c.d.)}
    \framesubtitle{Funkcje składowe}

    Deklaracja funkcji wewnątrz definicji struktury, definicja funkcji poza:

    {\scriptsize
    \begin{lstlisting}
    $// Cat.h$
    struct Cat {
        auto make_sound() const -> void;
    };

    $// Cat.cpp$
    auto Cat::make_sound() const -> void
    {
        std::cout << ("Meow! (" + name + ")\n");
    }
    \end{lstlisting}}

    Wewnątrz definicji funkcji można używać jej pól (stała składowa {\tt name}
    na przykładzie).
\end{frame}

\section{Zadania}

\begin{frame}
    \frametitle{Zadanie: student}
    \label{lecture_exercise_0}

    Zaimplementować strukturę danych opisującą studenta. Struktura powinna
    składać się z:
    \begin{enumerate}
        \item pól (imię, nazwisko, numer indeksu, aktualny semest,
            średnia ocen)
        \item funkcji składowej `{\tt to\_string() const}' zwracającej {\tt
            std::string}, którym opisuje studenta
        \item konstruktora
    \end{enumerate}

    Niech w funkcji main będzie utworzony obiekt reprezentujący was, a na
    \texttt{std::cout} wydrukowany będzie wynik działania funkcji
    \texttt{Student::to\_string} na tym obiekcie.

    \vspace{1em}

    {\footnotesize
    Kod źródłowy w plikach {\tt include/s1234/Student.h} (nagłówek) i
    {\tt src/s03-Student.cpp} (implementacja i funkcja {\tt main}).}
\end{frame}

\begin{frame}
    \frametitle{Zadanie: czas}
    \label{lecture_exercise_1}

    Zaimplementować strukturę danych opisującą czas. Struktura powinna składać
    się z:
    \begin{enumerate}
        \item pól (godzina, minuta, sekunda)
        \item funkcji składowych:
            \begin{enumerate}
                \item `{\tt to\_string() const}' zwracającej {\tt
                    std::string} pokazującej czas w formacie {\tt HH:MM:SS}
                \item {\tt next\_hour()}, {\tt next\_minute()}, i {\tt
                    next\_second()} (wszystkie zwracające {\tt void})
                    zwiększających czas
            \end{enumerate}
        \item konstruktora
    \end{enumerate}

    Niech w funkcji \texttt{main} pojawi się kod pozwalający na zweryfikowanie
    działania waszej struktury danych dla godziny 23:59:59 (np. niech drukuje
    godzinę, zwiększy mintę, wydrukuje znowu, itd.).

    \vspace{1em}

    {\footnotesize
    Kod źródłowy w plikach {\tt include/s1234/Time.h} (nagłówek) i
    {\tt src/s03-Time.cpp} (implementacja i funkcja {\tt main}).}
\end{frame}

\begin{frame}
    \frametitle{Zadanie: pora dnia}
    \label{lecture_exercise_2}

    Do struktury opisującej czas dodać funkcję składową
    `{\tt time\_of\_day() const}' zwracającą porę dnia (rano, dzień, wieczór,
    noc). Pora dnia powinna być opisana typem wyliczeniowym {\tt enum class
    Time\_of\_day}.

    Napisać funkcję \texttt{to\_string(Time\_of\_day)} zwracającą
    \texttt{std::string}, która zamieni wartość wyliczeniową na napis.

    W funkcji main dodać kod pozwalający na zweryfikowanie działania dodanych
    funkcji.

    \vspace{1em}

    {\footnotesize
    Kod źródłowy w plikach {\tt include/s1234/Time.h} (nagłówek) i
    {\tt src/s03-Time.cpp} (implementacja i funkcja {\tt main}).}
\end{frame}

\begin{frame}[fragile]
    \frametitle{Zadanie: arytmetyka}
    \label{lecture_exercise_3}

    Do struktury opisującej czas dodać funkcje składowe:

    {\scriptsize
    \begin{lstlisting}
    auto operator+ (Time const&) const -> Time;
    auto operator- (Time const&) const -> Time;
    auto operator< (Time const&) const -> bool;
    auto operator> (Time const&) const -> bool;
    auto operator== (Time const&) const -> bool;
    auto operator!= (Time const&) const -> bool;
    \end{lstlisting}}

    Umożliwią one arytmetykę (dodawanie, odejmowanie, porównywanie, itd.) na czasie.

    Do funkcji \texttt{main} dodać kod, który pozwoli na zweryfikowanie
    działania.

    \vspace{1em}

    {\footnotesize
    Kod źródłowy w plikach {\tt include/s1234/Time.h} (nagłówek) i
    {\tt src/s03-Time.cpp} (implementacja i funkcja {\tt main}).}
\end{frame}

\begin{frame}[fragile]
    \frametitle{Zadanie: sekundy do północy}
    \label{lecture_exercise_4}

    Do struktury opisującej czas dodać funkcje składowe:

    {\scriptsize
    \begin{lstlisting}
    auto count_seconds() const -> uint64_t;
    auto count_minutes() const -> uint64_t;
    auto time_to_midnight() const -> Time;
    \end{lstlisting}}

    Funkcje \texttt{count\_seconds()} i \texttt{count\_minutes()} liczą sekundy
    \emph{od} godziny 00:00:00.\\
    Funkcja \texttt{time\_to\_midnight()} zwraca czas pozostały \emph{do} północy.

    Do funkcji \texttt{main} dodać kod, który pozwoli na zweryfikowanie
    działania.

    \vspace{1em}

    {\footnotesize
    Kod źródłowy w plikach {\tt include/s1234/Time.h} (nagłówek) i
    {\tt src/s03-Time.cpp} (implementacja i funkcja {\tt main}).}
\end{frame}

%% zewnętrzne
%% składowe

\section{OOP}

\begin{frame}
    \frametitle{OOP}

    OOP -- programowanie zorientowane obiektowo (ang. \emph{object-oriented
    programming}).

    \vspace{1em}

    Na początku lat 70. XX wieku powstał język
    Smalltalk\footnote{\url{https://en.wikipedia.org/wiki/Smalltalk}}. Jego
    głównym zamysłem była swobodna \emph{wymiana wiadomości} między
    \emph{obiektami} (dlatego ''zorientowane obiektowo'') współistniejącymi w
    systemie.

    W późnych latach 80. XX wieku powstał język
    Erlang\footnote{\url{https://en.wikipedia.org/wiki/Erlang_(programming_language)}},
    którego jednym z głównych mechanizmów była \emph{asynchroniczna} wymiana
    wiadomości.
\end{frame}

\begin{frame}
    \frametitle{OOP (c.d.)}

    Na początku lat 60. XX wieku został opracowany język
    Simula\footnote{\url{https://en.wikipedia.org/wiki/Simula}}, który z językiem
    Smalltalk nie miał zbyt wiele wspólnego, ale za to posiadał klasy i
    dziedziczenie.

    \vspace{1em}

    W 1985 powstał język C++\footnote{\url{https://en.wikipedia.org/wiki/C++}},
    który zmiksował składnię i funkcjonalność języka C z mechanizmami języka
    Simula (klasy i dziedziczenie).
\end{frame}

\begin{frame}
    \frametitle{OOP (c.d.)}

    W 1995 powstał język
    Java\footnote{\url{https://en.wikipedia.org/wiki/Java_(programming_language)}},
    który wzorował się na Simuli i C++ - też miał klasy i dziedziczenie.
    W Java wszystkie klasy mają wspólną klasę bazową - nazwaną, pechowo, {\tt
    Object}. Nie jest to jedyna pechowa decyzja w tym języku...

    \vspace{1em}

    Dział marketingu Oracle uznał, że to oznacza, że język jest zorientowany
    obiektowo i od tamtego momentu OOP zmieniło się nie do poznania\footnote{na
    zajęciach OOP będzie oznaczać tą drugą, zdegenerowaną formę, a nie
    oryginalną ideę}.\\Zamiast oznaczać swobodną, asynchroniczną wymianę
    wiadomości zaczęło oznaczać programowanie oparte na klasach i wzorce
    projektowe usiłujące zamaskować biedę semantyczną języka Java.
\end{frame}

\subsection{...w C++}

\begin{frame}
    \frametitle{Dziedziczenie -- po co?}
    \framesubtitle{OOP}

    Dziedziczenie (ang. \emph{inheritance}) polega na zbudowaniu jednego typu
    (\emph{pochodnego}; ang.  \emph{derived type}), w oparciu o drugi typ
    (\emph{bazowy}; ang. \emph{base type}). Mówimy, że typ pochodny
    \emph{dziedziczy po} typie bazowym.

    \vspace{1em}

    Typ pochodny posiada ''z automatu'' wszystkie pola i funkcje składowe typu
    bazowego, ale dostęp ma tylko publicznych\footnote{Tak naprawdę ma jeszcze
    dostęp do pól chronionych (słowo kluczowe {\tt protected}), ale pola
    chronione to coś czego nie polecam używać. Dziedziczenie implementacji jest
    brudnym mechanizmem nawet bez dodatkowych komplikacji.}, czyli typ pochodzny
    \emph{dziedziczy implementację}\footnote{To czy odziedziczone pola są
    widoczne ''na zewnątrz'' typu pochodnego zależy od tego czy dziedziczył
    prywatnie (wtedy nie są) czy publicznie (wtedy są).} typu bazowego.
\end{frame}

\begin{frame}
    \frametitle{Polimorfizm -- po co?}
    \framesubtitle{OOP}

    Polimorfizm (ang. \emph{polymorphism}) polega na wykorzystaniu wspólnego
    interfejsu dla wielu typów. Typy prezentujące ten interfejs mogą być używane
    zamiennie.

    \vspace{1em}

    Polimorfizm w odniesieniu do funkcji oznacza, że może ona działać na wielu
    typach.\footnote{Jest to \emph{de facto} druga strona tego co powiedzieliśmy
    o typach, bo można powiedzieć, że funkcja akceptuje typy spełniające pewien
    interfejs i w ten sposób zatoczyć koło.}

    \vspace{1em}

    W C++ żeby móc wykorzystać polimorfizm potrzebe są wskaźniki lub referencje
    (na etapie działania programu, ang. \emph{run time}), lub szablony (na
    etapie kompilacji, ang. \emph{compile time}).
\end{frame}

\begin{frame}
    \frametitle{Po co? (moim zdaniem)}
    \framesubtitle{OOP}

    Dziedziczenie w C++ jest sposobem na reprezentację relacji między typami
    danych. Czyli jego \emph{pierwszorzędną} rolą jest zapewnienie hierarchii
    typów, a dopiero \emph{drugorzędną} (niemal przy okazji) dziedziczenie
    implementacji.

    % \vspace{1em}

    % Polimorfizm w C++ jest sposobem na podanie typu \emph{pochodnego} jako argumentu
    % tam, gdzie parametr jest zadeklarowany jako typ \emph{bazowy}.

    \vspace{1em}

    Najprościej jak się da, nie ma co sobie komplikować życia. Jeśli chcemy
    współdzielić implementację mamy do tego dedykowany mechanizm -
    \emph{procedural abstraction}\footnote{patrz pierwszy wykład} czyli funkcje.
\end{frame}

%% po co dziedziczenie?
%%      - ontological inheritance, specialisation: T is a specific kind of B
%%      - abstract data type inheritance, substitution: T behaves as if it was B
%%      - implementation inheritance, code sharing: T has all properties of B,
%%        but augments them in certain ways

\section{Zadania (OOP)}

\begin{frame}
    \frametitle{Zadanie: kalkulator}
    \label{lecture_exercise_5}

    Rozwinąć kalkulator napisany w stylu obiektowym tak, żeby nie odbiegał
    funkcjonalnością od kalkulatora napisanego w stylu proceduralnym
    zaimplementowanego w ramach poprzednich zajęć.

    Kod do wystartowania znajduje się w repozytorium z
    materiałami\footnote{\url{https://git.sr.ht/~maelkum/education-introduction-to-programming-cxx}}
    w plikach \texttt{src/04-rpn-calculator-oo.cpp} (implementacja) i
    \texttt{include/RPN\_calculator.h} (plik nagłówkowy).

    \vspace{1em}

    {\footnotesize
    Wasz kod powinien znaleźć się w plikach
    \texttt{src/s04-rpn-calculator-oo.cpp} (implementacja) i
    \texttt{include/RPN\_calculator.h} (plik nagłówkowy).}
\end{frame}

\section{Podsumowanie}

\begin{frame}
    \frametitle{Co nowego?}
    \frametitle{Podsumowanie}

    Student powinien umieć:

    \begin{enumerate}
        \item samodzielnie zaprojektować własny typ danych, jego pola i funkcje
            składowe
        \item wytłumaczyć czym jest i jak działa funkcja składowa, oraz czym
            jest {\tt this}
        \item powiedzieć jaka jest rola konstruktora
    \end{enumerate}
\end{frame}

\begin{frame}
    \frametitle{Zadania}
    \framesubtitle{Podsumowanie}

    Zadania znajdują się na slajdach
    \ref{lecture_exercise_0},
    \ref{lecture_exercise_1},
    \ref{lecture_exercise_2},
    \ref{lecture_exercise_3},
    \ref{lecture_exercise_4},
    \ref{lecture_exercise_5}.
\end{frame}

\end{document}
