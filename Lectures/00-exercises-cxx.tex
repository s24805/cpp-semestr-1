\documentclass[11pt,a4paper,titlepage,onecolumn]{article}

\usepackage[utf8]{inputenc}
\usepackage{textcomp}
\usepackage[official]{eurosym}
\usepackage[polish]{babel}
\usepackage{amsthm}
\usepackage{graphicx}
\usepackage[T1]{fontenc}
\usepackage{scrextend}
\usepackage{hyperref}
\usepackage{xcolor}
\usepackage[inline]{enumitem}
% \usepackage{nameref}
% \usepackage{showlabels}
% \usepackage{titlesec}
\usepackage{geometry}
\usepackage{rotating}
\geometry{a4paper, portrait, margin=2cm}
\usepackage{listings}
\usepackage{tabularx}
\usepackage{longtable}
\usepackage[export]{adjustbox}

\setcounter{secnumdepth}{4}

\renewcommand*{\lstlistlistingname}{Spis listingów}

\author{Marek Marecki}
\title{Podstawy programownia (w języku C++)\\{\Large Ćwiczenia}}

\definecolor{light-gray}{gray}{0.9}

\lstset{basicstyle=\ttfamily\color{black},
columns=fixed,
escapeinside={[*}{*]},
inputencoding=utf8,
extendedchars=true,
moredelim=**[is][\color{red}]{@}{@},
moredelim=**[is][\color{gray}]{`}{`},
moredelim=**[is][\color{olive}]{$}{$}}

\begin{document}

\maketitle

% \frontmatter
\tableofcontents
% \listoftables
% \listoffigures
\lstlistoflistings
% \vspace*{\fill}

\newpage

% \part{X}
% \section{X}
% \subsection{X}
% \subsubsection{X}

\section{Podstawy}

\begin{figure}[!htp]
\begin{lstlisting}[caption={Hello, World!},
    captionpos=b,
    label=listing_0000,
    language=c]
#include <iostream>

auto main() -> int
{
    std::cout << "Hello, World!\n";
    return 0;
}
\end{lstlisting}
\end{figure}

\paragraph{Hello, World!} Zmodyfikuj program z listingu
\ref{listing_0000} tak żeby wyświetlał twoje imię i nazwisko, lub
jakiś inny wybrany tekst.

\begin{figure}[!htp]
\begin{lstlisting}[caption={Hello, World!},
    captionpos=b,
    label=listing_0001,
    language=c]
#include <iostream>
#include <string>

auto ask_user_for_integer(std::string const prompt) -> int
{
    if (not prompt.empty()) {
        std::cout << prompt;
    }
    auto value = std::string{};
    std::getline(std::cin, value);
    return std::stoi(value);
}
\end{lstlisting}
\end{figure}

\paragraph{Dodawanie} Wykorzystując funkcję z listingu \ref{listing_0001} napisz
program, który pobierze od użytkownika dwie liczby i doda je do siebie. Wynik
wydrukuj na \texttt{std::cout}.

\paragraph{Mnożenie} Wykorzystując funkcję z listingu \ref{listing_0001} napisz
program, który pobierze od użytkownika dwie liczby i pomnoży je przez siebie.
Wynik wydrukuj na \texttt{std::cout}.

\paragraph{Większa liczba} Wykorzystując funkcję z listingu \ref{listing_0001} napisz
program, który pobierze od użytkownika dwie liczby i wydrukuje większą z nich.
Wynik wydrukuj na \texttt{std::cout}.

\paragraph{Wartość absolutna} Napisz program, który pobierze od użytkownika
liczbę i poda jej wartość absolutną. Wynik wydrukuj na \texttt{std::cout}.

\begin{figure}[!htp]
\begin{lstlisting}[caption={relacja między liczbami},
    captionpos=b,
    label=listing_0001_comparison]
$./program 2 2$
2 == 2
$./program 0 3$
0 < 3
$./program 1 -1$
1 > -1
\end{lstlisting}
\end{figure}

\paragraph{Relacja między liczbami}\label{ex_0001_relationship_between_numbers}
Wykorzystując funkcję z listingu \ref{listing_0001} napisz program, który
pobierze od użytkownika dwie liczby i wydrukuje relację między nimi tak jak na
listingu \ref{listing_0001_comparison}.
Wynik wydrukuj na \texttt{std::cout}.

\paragraph{Dodatnia-nieujemna-ujemna} Napisz program, który pobierze od
użytkownika liczbę i poda następujący wynik:
\begin{enumerate}
    \item \texttt{1} jeśli liczba jest dodatnia
    \item \texttt{0} jeśli liczba jest zerem
    \item \texttt{-1} jeśli liczba jest ujemna
\end{enumerate}
Wynik wydrukuj na \texttt{std::cout}.

\paragraph{Największa} Napisz program, który pobierze od
użytkownika trzy liczby i wydrukuje największą.
Wynik wydrukuj na \texttt{std::cout}.

%%%%%%%%%%%%%%%%%%%%%%%%%%%%%%%%%%%%%%%%%%%%%%%%%%%%%%%%%%%%%%%%%%%%%%%%%%%%%%%%
% Loops
%%%%%%%%%%%%%%%%%%%%%%%%%%%%%%%%%%%%%%%%%%%%%%%%%%%%%%%%%%%%%%%%%%%%%%%%%%%%%%%%
\newpage
\section{Pętle}

\paragraph{Lista liczb}\label{ex_0001_list_of_numbers} Napisz program, który pobierze od użytkownika dwie
liczby (\texttt{a} i \texttt{b}), a następnie wydrukuje listę liczb większych
lub równych \texttt{a} i mniejszych od \texttt{b}.
Wynik wydrukuj na \texttt{std::cout}.

\paragraph{Lista liczb (2)} Rozwiń program z poprzedniego zadania tak żeby
pobierał trzecią liczbę (\texttt{c}) i drukował jedynie liczby podzielne przez
\texttt{c}. Upewnij się, że program odrzuci \texttt{c} równe 0.
Wynik wydrukuj na \texttt{std::cout}.

\paragraph{Lista liczb (3)} Rozwiń program z zadania \ref{ex_0001_list_of_numbers} tak żeby
pobierał liczbę \texttt{s} i użył jej jako kroku pętli. Upewnij się, że program
działa też dla ujemnej liczby \texttt{s}. Upewnij się, że program odrzuci krok o
wartości 0.
Wynik wydrukuj na \texttt{std::cout}.

\paragraph{Liczba pierwsza} Napisz program, który pobierze od użytkownika liczbę
i sprawdzi czy jest ona liczbą pierwszą.
Wynik wydrukuj na \texttt{std::cout}.

\paragraph{Suma liczb pierwszych} Napisz program, który pobierze od użytkownika liczbę
i sprawdzi czy jest ona liczbą pierwszą. Jeśli tak, to niech poda sumę liczb
pierwszych mniejszych lub równych podanej liczbie.
Wynik wydrukuj na \texttt{std::cout}.

\begin{figure}[!htp]
\begin{lstlisting}[caption={relacja między liczbami (2)},
    captionpos=b,
    label=listing_0001_comparison_2]
$./program 2 2 0 3 8 -1$
2 == 2
2 > 3
2 < 3
2 < 8
2 > -1
\end{lstlisting}
\end{figure}

\paragraph{Relacja między liczbami (2)} Rozwiń program z zadania
\ref{ex_0001_relationship_between_numbers} tak, żeby porównywał więcej liczb
naraz, tak jak na listingu \ref{listing_0001_comparison_2}.
Wynik wydrukuj na \texttt{std::cout}.

\paragraph{Suma podzielnych} Napisz program, który pobierze od użytkownika dwie
liczby: limit i dzielnik. Niech program obliczy sumę wszystkich liczb większych
od zera, ale mniejszych lub równych \emph{limitowi}, które są podzielne przez
\emph{dzielnik}.
Wynik wydrukuj na \texttt{std::cout}.

\begin{figure}[!htp]
\begin{lstlisting}[caption={pętla for},
    captionpos=b,
    label=listing_0002]
for (auto i = 0; i < 42; ++i) {
    $// do something$
}
\end{lstlisting}
\end{figure}

\begin{figure}[!htp]
\begin{lstlisting}[caption={pętla while},
    captionpos=b,
    label=listing_0003]
auto i = 0;
while (i < 42) {
    $// do something$
    ++i;
}
\end{lstlisting}
\end{figure}

\begin{figure}[!htp]
\begin{lstlisting}[caption={pętla do-while},
    captionpos=b,
    label=listing_0004]
auto i = 0;
do {
    $// do something$
    ++i;
} while (i < 42);
\end{lstlisting}
\end{figure}

\paragraph{Silnia (for)} Wykorzystując pętlę \texttt{for} (patrz listing
\ref{listing_0002}) napisz program, który pobierze od użytkownika liczbę i
obliczy jej silnię. Wynik wydrukuj na \texttt{std::cout}.

\paragraph{Silnia (while)} Wykorzystując pętlę \texttt{while} (patrz listing
\ref{listing_0003}) napisz program, który pobierze od użytkownika liczbę i
obliczy jej silnię. Wynik wydrukuj na \texttt{std::cout}.

\paragraph{Silnia (do-while)} Wykorzystując pętlę \texttt{do-while} (patrz listing
\ref{listing_0004}) napisz program, który pobierze od użytkownika liczbę i
obliczy jej silnię. Wynik wydrukuj na \texttt{std::cout}.

\begin{figure}[!htp]
\begin{lstlisting}[caption={prostokąt z gwiazdek},
    captionpos=b,
    label=listing_0005]
$./program-prostokat 2 4$
****
****
\end{lstlisting}
\end{figure}

\begin{figure}[!htp]
\begin{lstlisting}[caption={trójkąt gwiazdek},
    captionpos=b,
    label=listing_0006]
$./program-odwrocony-trojkat 4$
*
**
***
****
\end{lstlisting}
\end{figure}

\begin{figure}[!htp]
\begin{lstlisting}[caption={odwrócony trójkąt gwiazdek},
    captionpos=b,
    label=listing_0007]
$./program-odwrocony-trojkat 4$
****
***
**
*
\end{lstlisting}
\end{figure}

\paragraph{Rysowanie figury (prostokąt)}
\label{ex_1_draw_a_rectangle_on_stdout}
Wykorzystując dowolną pętlę napisz
program, który pobierze z wiersza poleceń wymiary prostokąta i narysuje go.
Wynik wydrukuj na \texttt{std::cout}. Przykładowe uruchomienie na listingu
\ref{listing_0005}.

\paragraph{Rysowanie figury (trójkąt)} Wykorzystując dowolną pętlę napisz
program, który pobierze z wiersza poleceń wymiary trójkąta i narysuje go. Wynik
wydrukuj na \texttt{std::cout}. Przykładowe uruchomienie na listingu
\ref{listing_0006}.

\paragraph{Rysowanie figury (odwrócony trójkąt)} Wykorzystując dowolną pętlę
napisz program, który pobierze z wiersza poleceń wymiary ,,odwróconego
trójkąta'' (tj.  niech wierzchołek będzie na dole, patrz listing
\ref{listing_0007}) i narysuje go. Wynik wydrukuj na \texttt{std::cout}.

\begin{figure}[!htp]
\begin{lstlisting}[caption={pusty kwardat},
    captionpos=b,
    label=listing_0008]
$./program-pusty-kwadrat 4$
****
*  *
*  *
****
\end{lstlisting}
\end{figure}

\paragraph{Rysowanie figury (pusty kwadrat)} Napisz program, który pobierze z
wiersza poleceń wymiary figury, a potem narysuje ,,pusty kwardat'' (patrz
listing \ref{listing_0008}). Wymiar nie może być mniejszy niż 3.
Wynik wydrukuj na \texttt{std::cout}.

%%%%%%%%%%%%%%%%%%%%%%%%%%%%%%%%%%%%%%%%%%%%%%%%%%%%%%%%%%%%%%%%%%%%%%%%%%%%%%%%
% Pointers
%%%%%%%%%%%%%%%%%%%%%%%%%%%%%%%%%%%%%%%%%%%%%%%%%%%%%%%%%%%%%%%%%%%%%%%%%%%%%%%%
\newpage
\section{Wskaźniki}

\begin{lstlisting}[caption={pobranie wskaźnika},
    captionpos=b,
    label=listing_howto_address_of]
auto x  = int{42};  $// an integer$
auto xp = &x;       $// a pointer to an integer (new style)$
int* xo = &x;       $// a pointer to an integer (old style)$
\end{lstlisting}

\begin{lstlisting}[caption={dereferencja wskaźnika},
    captionpos=b,
    label=listing_howto_pointer_dereference]
auto x = int{42};
auto xp = &x;   $// xp contains the address of the x variable$
auto xd = *xp;  $// xd contains a copy of the variable to which
                // xp is pointing$

*xp = 64;       $// x contains 64 after this assignment$
\end{lstlisting}

\paragraph{Pobranie wskaźnika} Napisz program, w którym w funkcji
\texttt{main()} utworzysz zmienną typu \texttt{std::string}, której wartością
będzie \texttt{Hello, World!}. Pobierz wskaźnik i wydrukuj adres tej zmiennej w
pamięci.\\
Wynik wydrukuj na \texttt{std::cout}.

\paragraph{Dereferencja}  Napisz funkcję \texttt{print()}, która będzie
jako parametr przyjmować wskaźnik na \texttt{std::string}. W funkcji
\texttt{print()} wydrukuj adres, na który wskazuje wskaźnik oraz napis stojący
za tym wskaźnikiem, np. ,,\texttt{1781f89a980 = Hello, World!}''. W funkcji
\texttt{main()} napisz kod, który wywołuje funkcję \texttt{print()}.
Wynik wydrukuj na \texttt{std::cout}.

\begin{figure}[!htp]
\begin{lstlisting}[caption={zamiana},
    captionpos=b,
    label=listing_0009_swap]
$./program-zamiana$
42 64
64 42
\end{lstlisting}
\end{figure}

\paragraph{Zamiana} Napisz funkcję \texttt{swap()}, która będzie
jako parametr przyjmować dwa wskaźniki na \texttt{int}.

W funkcji \texttt{main()} napisz kod, który wywołuje funkcję \texttt{swap()}.
Wydrukuj wartość dwóch testowych liczb przed i po zamianie (patrz listing
\ref{listing_0009_swap}).

Wynik wydrukuj na \texttt{std::cout}.

\paragraph{\texttt{memset(3)}} Zaimplementuj funkcję \texttt{memset(3)}. Jej
opis znajduje się w podręczniku użytkownika systemu. Można go wyświetlić
używając polecenia `\texttt{man 3 memset}'.

\paragraph{\texttt{memcpy(3)}} Zaimplementuj funkcję \texttt{memcpy(3)}.

\paragraph{\texttt{memfrob(3)}} Zaimplementuj funkcję \texttt{memfrob(3)}.

\paragraph{\texttt{memrev()}} Zaimplementuj funkcję \texttt{memrev(void* s,
size\_t n)}, która odwróci kolejność bajtów w obszarze pamięci o rozmiarze
\texttt{n}, na który wskazuje wskaźnik \texttt{s}.

\paragraph{\texttt{memrand()}} Zaimplementuj funkcję \texttt{memrev(void* s,
size\_t n)}, która wypełni losowymi bajtami obszar pamięci o rozmiarze
\texttt{n}, na który wskazuje wskaźnik \texttt{s}.

%%%%%%%%%%%%%%%%%%%%%%%%%%%%%%%%%%%%%%%%%%%%%%%%%%%%%%%%%%%%%%%%%%%%%%%%%%%%%%%%
% Struktury danych
%%%%%%%%%%%%%%%%%%%%%%%%%%%%%%%%%%%%%%%%%%%%%%%%%%%%%%%%%%%%%%%%%%%%%%%%%%%%%%%%
\newpage
\section{Struktury danych}

Dla każdego zadania powinna być napisana funkcja \texttt{main()}, która będzie
testować działanie zaimplementowanej struktury (lub struktur) danych wywołując
jej funkcje składowe.

O ile nie jest powiedziane inaczej, w funkcji \texttt{main()} dane (np. długości
boków, ilości żyć, itd.) mogą być wpisane ,,na sztywno'' podczas tworzenia
obiektów danej struktury danych i nie muszą być wczytywane od użytkownika.

\begin{figure}[!htp]
{\small
\begin{lstlisting}[caption={struct},
    captionpos=b,
    label=listing_0010_struct]
struct A_type {
    std::string member_variable;
    int const member_constant;

    auto member_function() -> std::string;
};

auto A_type::member_function() -> std::string
{
    return (member_variable
            + ": "
            + std::to_string(member_constant));
}
\end{lstlisting}}
\end{figure}

\begin{figure}[!htp]
{\small
\begin{lstlisting}[caption={konstruktor},
    captionpos=b,
    label=listing_0011_ctor]
struct @Foo@ {
    std::string const bar;

    $// ctor's name must be the same as struct's name$
    @Foo@(std::string);
};

@Foo@::@Foo@(std::string b)
    : bar{std::move(b)}
{}
\end{lstlisting}}
\end{figure}

\paragraph{Kwadrat} Zaprojektur strukturę danych reprezentującą kwadrat. Niech
posiada ona jedno stałe pole typu \texttt{float}, które będzie reprezentowało
długość boku kwadratu. Wartość musi być inicjalizowana w konstruktorze.

Struktura ma posiadać dwie funkcje składowe:
\begin{enumerate}
    \item \texttt{auto area() const -> float} zwracającą pole kwadratu
    \item \texttt{auto draw() const -> void} rysującą kwadrat na ekranie (tak
        jak w zadaniu \ref{ex_1_draw_a_rectangle_on_stdout})
\end{enumerate}

\paragraph{Prostokąt} Zaprojektur strukturę danych reprezentującą prostokąt.
Niech ma ona dwa stałe pola reprezentujące długości dwóch boków prostokąta.

Struktura ma posiadać dwie funkcje składowe:
\begin{enumerate}
    \item \texttt{auto area() const -> float} zwracającą pole prostokąta
    \item \texttt{auto draw() const -> void} rysującą prostokąt na ekranie (tak
        jak w zadaniu \ref{ex_1_draw_a_rectangle_on_stdout})
\end{enumerate}

\paragraph{Prostokąt (2)} Zmodyfikuj kod poprzedniego zadania tak, aby możliwe
było dodanie funkcji składowej \texttt{auto scale(float const x, float const y) -> void}.
Funkcja \texttt{resize()} ma skalować rozmiar boków prostokąta przez pewien
mnożnik.\\
Przykłady:
\begin{enumerate}
    \item \texttt{rect.scale(1.5f, 1.5f)} wydłuży oba boki prostokąta o 50\%
    \item \texttt{rect.scale(2.0f, 0.5f)} wydłuży jeden bok prostokąta
        dwukrotnie, a drugi skróci dwukrotnie
\end{enumerate}


% \newpage

% \mainmatter
% \part{Cel, strategia wykonania, i przebieg prac}

% \input{wprowadzenie/main}

% \input{strategia/main}

% \input{przebieg/main}

% \part{Język \ViuAct\ i jego kompilator}

% \input{viuact/zal/main}

% \input{viuact/spec/main}

% \input{viuact/impl/main}

% \part{Program ViuaChat}

% \input{chat/zal/main}

% \input{chat/impl/main}

% \part{Podsumowanie}

% \input{raport/main}

% \input{wklad_wlasny/main}

% \bibliographystyle{ieeetr}
% \bibliographystyle{apalike}
% \bibliographystyle{acm}
% \bibliographystyle{alpha}
% \bibliography{bibliografia}

% \part{Załączniki}

% \appendix
% \input{appendix/viua_vm_asm_language/main}
% \input{appendix/viua_vm_execution_model/main}
% \input{appendix/plain_websocket/main}
% \input{appendix/issue_tracking/main}

\end{document}
