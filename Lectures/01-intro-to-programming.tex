\documentclass[aspectratio=169]{beamer}

\usepackage[utf8]{inputenc}
\usepackage{textcomp}
\usepackage[polish]{babel}
\usepackage{amsthm}
\usepackage{graphicx}
\usepackage[T1]{fontenc}
\usepackage{scrextend}
\usepackage{hyperref}
\usepackage{xcolor}
\usepackage{geometry}
\usepackage{listings}
\usepackage{epigraph}

\renewcommand{\epigraphsize}{\scriptsize}

\usetheme{-bjeldbak/beamerthemebjeldbak}

\definecolor{xbarcolor}{HTML}{0f2f0f}
\setbeamercolor{lower separation line head}{bg=xbarcolor} 
\setbeamercolor{lower separation line foot}{bg=xbarcolor} 

\title{Podstawy programownia (w języku C++)}
\subtitle{Wstęp do programowania}
\author{Marek Marecki}
\institute{Polsko-Japońska Akademia Technik Komputerowych}

\lstset{basicstyle=\ttfamily\color{black},
columns=fixed,
escapeinside={\%*}{*)},
inputencoding=utf8,
extendedchars=true,
moredelim=**[is][\color{red}]{@}{@}}

\begin{document}

{%
    \setbeamertemplate{headline}{}
    \frame{\titlepage}
}

\section{Rys historyczny}

\begin{frame}
    \frametitle{Środek historii}
    \framesubtitle{...czyli 150 lat do CPU}

    {\small
    \begin{labeling}{1822}
    \item[1822] Difference Engine\footnote{{\tiny skonstruowany w londyńskim
        Science Museum w 1991}} -- Charles Babbage
    \item[1837] Analytical Engine -- Charles Babbage hardware\footnote{\tiny
        zbudował prototyp CPU w 1871} \footnote{{\tiny w 1906 jego syn, Henry Babbage,
        zbuduje kompletene CPU}}, a Ada Lovelace software
    \item[1941] Konrad Zuse -- Z3, pierwszy programowalny
        komputer\footnote{{\tiny zniszczony podczas bombardowania Berlina przez
        Aliantów}}
    \item[1944] Harvard Mark I -- drugi programowalny komputer
    \item[1971] Intel 4004 -- 4-bitowy mikroprocesor
    \end{labeling}}

    W 2016 UK pozwoliło sprzedać ARM -- czyli po raz drugi wypuścili z rąk ważny
    kawałek technologii.
\end{frame}

\begin{frame}
    \frametitle{Architektury CPU}

    \begin{labeling}{1837}
        \item[1837] Analytical Engine
        \item[1978] x86
        \item[1985] ARM, MIPS
        \item[1991] PowerPC
        \item[2001] Itanium {\tiny (VLIW; \emph{failed})}
        \item[2003] Mill {\tiny (VLIW; \emph{in development})}
        \item[2010] RISC-V
    \end{labeling}
\end{frame}

\section{Wstęp do komputerów}

\begin{frame}
    \frametitle{von Neumann}

    \begin{enumerate}
        \item CPU
        \item RAM
        \item pamięć masowa
        \item I/O
    \end{enumerate}
\end{frame}

\section{Wstęp do języków programowania}

\begin{frame}
    \frametitle{A co to?}

    Sposób na wyrażenie swoich żądań względem maszyny.

    \vspace{1em}

    Kontrakt z demonem -- spełnia rozkazy \emph{dokładnie tak jak są
    wypowiedziane}, bez oglądania się na \emph{intencje} programisty.
\end{frame}

\begin{frame}
    \frametitle{A po co to komu?}
    {\tt mov eax, 0x2a} \textsubscript{(x86)}\\
    vs\\
    {\tt auto x = int\{42\};} \textsubscript{(C++)}

    \vspace{2em}

    Dużo prostsze wydawanie maszynie skomplikowanych rozkazów, i łatwość
    zrozumienia znaczenia programu.\\
    Automatyczna alokacja rejestrów i pamięci; automatyczne skoki; ergonomiczna
    semantyka.

    \vspace{1em}

    Przenośność (ang. \emph{portability}) programów między platformami.
\end{frame}

\begin{frame}
    \frametitle{Umowny podział}

    \begin{enumerate}
        \item compiled \emph{vs} interpreted (JIT?)
        \item typing: static \emph{vs} dynamic, strong \emph{vs} weak
        \item paradigm\footnote{Seven Languages in Seven Weeks; Bruce A. Tate;
            ISBN-13: 978-1-934356-59-3}: functional \emph{vs} object-oriented
            \emph{vs} structural \emph{vs} prototype-based \emph{vs} ...
        \item rodziny: C-like (pochodne po języku ALGOL), ML-like, Lisp-like
        \item ''toy'' \emph{vs} ''real''
    \end{enumerate}

    \vspace{1em}

    \begin{labeling}{Smalltalk}
        \item[C] compiled, static-weak typing, structural
        \item[C++] compiled, static-strong typing, multiparadigm
        \item[Smalltalk] interpreted, dynamic-strong, object-oriented
        \item[OCaml] compiled, static-strong, functional
        \item[Perl] interpreted, dynamic-weak, multiparadigm
    \end{labeling}
\end{frame}

\section{Składniki języka}

\begin{frame}
    \frametitle{Co jest potrzebne w języku?}

    ?{\tiny \textsubscript{(pytanie do sali)}}
\end{frame}

\begin{frame}
    \frametitle{Z punktu widzenia prostego człowieka}

    \begin{enumerate}
        \item \emph{control flow} -- mechanizmy przepływu kontroli, czyli
            sterowania programem
        \item \emph{data structures} -- reprezentacja struktur danych
        \item \emph{I/O} -- zapis i odczyt danych, czyli sposób na interakcję ze
            światem zewnętrznym
    \end{enumerate}
\end{frame}

\begin{frame}
    \frametitle{Jackson Structured Programming}
    \framesubtitle{Control flow}

    Michael Jackson, 1975; Principles of Program Design

    \vspace{1em}

    \begin{enumerate}
        \item \emph{sequence} -- sekwencjonowanie, czyli ustalenie kolejności
            wykonywania operacji
        \item \emph{selection} (\emph{alternative}) -- wybór (alternatywa),
            czyli decyzja o podjęciu jednej z kilku różnych ścieżek wykonania
        \item \emph{iteration} -- iteracja, czyli powtarzanie tych samych kroków
            $n$ razy
    \end{enumerate}

    \vspace{1em}

    Nadaje się do opisu algorytmów, ale nie za bardzo do czegoś więcej. Często
    tak jest z różnymi modelami -- są wygodne w teorii, ale niezbyt praktyczne.
\end{frame}

\begin{frame}
    \frametitle{Warnier/Orr}
    \framesubtitle{Control flow}

    Jean-Dominique Warnier, 1976; Logical construction of programs\\
    Kenneth Orr, 1977; Structured systems development

    \vspace{1em}

    \begin{enumerate}
        \item \emph{recursion} -- rekurencja, czyli sposób na zagnieżdżone
            wykonywanie operacji
        \item \emph{concurrency} -- współbieżność, czyli sposób na wykonywanie
            kilku operacji ''w tym samym czasie'' (naprzemiennie na jednym
            procesorze, lub równolegle\footnote{ten wariant nazywa się
            \emph{parallelism}, i czasem jest podawany obok współbieżności jako
            coś innego} na wielu)
    \end{enumerate}

    \vspace{1em}

    Rekurencja i współbieżność są nieodłącznymi elementami programów, które
    działają w ''prawdziwym świecie''. Bez nich niemożliwe byłoby interaktywne
    używanie komputerów.
\end{frame}

\begin{frame}
    \frametitle{Michael Scott}
    \framesubtitle{Control flow}

    Michael Lee Scott, 2000; Programming language pragmatics\footnote{ISBN
    1-55860-442-1}

    \vspace{1em}

    \begin{enumerate}
        \item \emph{procedural abstraction} -- zbiór operacji opakowany w sposób
            umożliwiający ich wspólne wywołanie, w skrócie: funkcja
        \item \emph{nondeterminacy} -- niedeterminizm, czyli sposób na
            zapewnienie losowości przy wyborze ściezki wykonania
        \item \emph{exceptions*} -- wyjątki, pozwalające na ''skok'' kontroli w
            przypadku wystąpienia błędu
    \end{enumerate}

    \vspace{1em}

    Funkcje i niedeterminizm zamykają bazowe mechanizmy, które służą kontroli
    przepływu.
\end{frame}

\begin{frame}
    \frametitle{Podsumowanie}
    \framesubtitle{Control flow}

    \begin{enumerate}
        \item \emph{sequence}
        \item \emph{selection}
        \item \emph{iteration}
        \item \emph{recursion}
        \item \emph{concurrency}
        \item \emph{procedural abstraction}
        \item \emph{nondeterminism}
        \item \emph{exceptions*}
    \end{enumerate}

    Egzotyczne metody kontroli przepływu -- \emph{continuations},
    \emph{coroutines}.
\end{frame}

\begin{frame}
    \frametitle{Bit, nibble\footnote{połowa bajtu, czyli 4 bity}, byte, word,
        half-word, double-word, quad-word...}
    \framesubtitle{Data structures}

    \epigraph{Na początku było słowo}{J 1,1-3}

    Bit - czyli wartość mogąca przechowywać $0$ lub $1$.

    Podstawową jednostką danych obsługiwanych przez CPU jest ''słowo'' -
    sekwencja bitów o pojedynczego długości rejestru. Dla architektury x86-64
    długość słowa to 64 bity.

    Zapis i odczyt słowa w pamięci zazwyczaj są \emph{operacjami atomowymi} co
    ma znaczenie dla programowania współbieżnego.
\end{frame}

\begin{frame}
    \frametitle{Liczby całkowite, ułamki, wartości logiczne}
    \framesubtitle{Data structures}

    Liczby całkowite - ze znakiem ({\tt signed}), bez znaku ({\tt unsigned}).\\
    Liczby zmiennoprzecinkowe - pojedynczej precyzji, podwójnej precyzji.\\
    Wartości logiczne - prawda, fałsz.
\end{frame}

\begin{frame}
    \frametitle{Znaki, napisy}
    \framesubtitle{Data structures}

    Znaki\footnote{kiedyś były najczęśniej szerokości 1 bajty (ASCII), ale
    obecnie, od upowszechnienia się standardu Unicode, są zazwyczaj zmiennej
    długości (UTF-8)} - reprezentujące pojedynczy glif (literę, znak
    interpunkcyjny, itd.) lub symbol kontrolny.\\
    Napisy - reprezentujące sekwencje znaków (np. {\tt ''Hello, World!''}).
\end{frame}

\begin{frame}
    \frametitle{Lista, kolejka, stos, zbiór, drzewo, krotka...}
    \framesubtitle{Data structures}

    {\scriptsize
    \begin{enumerate}
        \item \emph{list} -- lista, czyli poszeregowana sekwencja wartości typu
            $T$, do których daje dostęp w dowolnej kolejności (ang.
            \emph{random~access})
        \item \emph{queue} -- kolejka, czyli poszeregowana sekwencja wartości
            typu $T$, do których daje dostęp na zasadzie \emph{FIFO}
        \item \emph{stack} -- stos, czyli poszeregowana sekwencja wartości
            typu $T$, do których daje dostęp na zasadzie \emph{LIFO}
        \item \emph{set} -- zbiór, nieposzeregowana kolekcja wartości typu $T$
        \item \emph{tree} -- drzewo, często wykorzystywane do budowania
            ''map'' czyli struktur asocjacyjnych pozwalających na przechowanie
            wartości typu $T$ pod kluczem typu $K$
        \item \emph{tuple} -- krotka, czyli struktura danych zawierająca $n$ pól
            typów $T_0$, $T_1$, ..., $T_n$
    \end{enumerate}}

    Podstawową złożoną strukturą danych jest \emph{tablica}, czyli sekwencja $n$
    elementów typu $T$. Bazując na tablicach bajtów (czyli de facto surowych,
    wydzielonych obszarach pamięci) można zaimplementować wszystkie powyższe
    struktury danych.
\end{frame}

\begin{frame}
    \frametitle{Typy użytkownika}
    \framesubtitle{Data structures}

    \begin{labeling}{\emph{struct}}
        \item[\emph{enum}] wyliczenie, czyli zdefiniowany zbiór wartości, które
            dany typ może przechować (vide \emph{sum~type})
        \item[\emph{struct}] struktura, czyli typ złożony z kilku pól różnych
                typów -- może przechowywać wszystkie kombinacje wartości pól
                (vide \emph{product~type})
    \end{labeling}

    Języki programowania często zapewniają programistom możliwość tworzenia
    własnych typów danych.
\end{frame}

\begin{frame}
    \frametitle{Wskaźniki i dynamiczna alokacja pamięci}
    \framesubtitle{Data structures}

    Implementacja wielu struktur danych (np. list lub napisów o zmiennej
    długości) byłaby niemożliwa bez wskaźników i dynamicznej alokacji pamięci.
\end{frame}

\begin{frame}
    \frametitle{Podsumowanie}
    \framesubtitle{Data structures}

    \begin{enumerate}
        \item słowo (\emph{word}), bajt (\emph{byte}), bit
        \item tablica (\emph{array})
        \item typy użytkownika (\emph{user-defined type} -- \emph{enum},
            \emph{struct})
        \item wskaźnik (\emph{pointer})
        \item dynamiczna alokacja pamięci (\emph{memory allocation})
        \item typy proste \emph{vs} typy złożone
    \end{enumerate}
\end{frame}

\begin{frame}
    \frametitle{Operacje wejścia-wyjścia}
    \framesubtitle{I/O}

    Sposób na interakcję i wymianę danych ze ''światem zewnętrznym'', czyli
    wszystkim tym co dzieje się poza CPU i pamięcią operacyjną (RAM).

    \begin{enumerate}
        \item \emph{I/O port}, \emph{MMU} (ang. \emph{memory management unit})
        \item \emph{file-descriptor}, \emph{socket} (ten sam interfejs dla
            plików i połączeń w sieci; POSIX)
        \item \emph{memory-mapped file}
    \end{enumerate}

    Zapisując i odczytując bajty da się obsłużyć każdy rodzaj urządzenia -
    monitor, klawiaturę, dysk, ramię robota, silnik, itd.
\end{frame}

\section{C++}

\begin{frame}
    \frametitle{Składniki języka C++}

    \begin{epigraphs}
    \qitem{Piękno, harmonia, wdzięk, dobry rytm - wszystkie zależą od
    prostoty.}{Platon, \emph{Republika}}
    \qitem{If you think it's simple, then you have misunderstood the
    problem.}{Bjarne Stroustrup, autor języka C++}
    \end{epigraphs}

    C++ jest rozbudowanym językiem oferującym wiele możliwości. Ceną za to jest
    jego ogromne skomplikowanie, brak ''elegancji'', i pułapki czychające na
    nieuważnego programistę.
\end{frame}

\begin{frame}
    \frametitle{Control flow - sequence}
    \framesubtitle{Składniki języka C++}

    Kontrola w języku C++ zaczyna się od pierwszej instrukcji w funkcji
    {\tt main()}. Kolejne instrukcje wykonywane są w kolejności zdefiniowanej w
    kodzie źródłowym i oddzielone operatorem \textbf{{\tt ;}}

    \vspace{1em}

    Grupy instrukcji są ograniczane nawiasami klamrowymi \textbf{{\tt \{ \}}} i
    traktowane jako pojedyncza (ale nie atomowa!) instrukcja.

    \vspace{1em}

    Wewnątrz instrukcji, kolejnością wykonania streuje również operator
    \textbf{{\tt ,}}
\end{frame}

\begin{frame}[fragile]
    \frametitle{Control flow - sequence}
    \framesubtitle{Składniki języka C++}

    \begin{lstlisting}
    auto x = 42;    // pierwsza instrukcja
    auto y = x;     // druga instrukcja
    print(y);       // trzecia instrukcja
    \end{lstlisting}
\end{frame}

\begin{frame}
    \frametitle{Control flow - selection}
    \framesubtitle{Składniki języka C++}

    \begin{labeling}{{\tt if-else}}
        \item[{\tt if-else}] wybiera następną instrukcję do wykonania na
            podstawie wartości logicznej dowolnego wyrażenia
        \item[{\tt switch}] wybiera nastepną instrukcję do wykonania na
            podstawie wartości typu wyliczeniowego (\emph{enum})
    \end{labeling}
\end{frame}

\begin{frame}[fragile]
    \frametitle{Control flow - selection ({\tt if})}
    \framesubtitle{Składniki języka C++}

    \begin{lstlisting}
    if (x < y) {
        std::cout << "x less than y\n";
    } else if (x > y) {
        std::cout << "x greater than y\n";
    } else if (x == y) {
        std::cout << "x equals y\n";
    } else {
        std::cout << "something else\n";
    }
    \end{lstlisting}
\end{frame}

\begin{frame}[fragile]
    \frametitle{Control flow - selection ({\tt switch})}
    \framesubtitle{Składniki języka C++}

    \begin{lstlisting}
    switch (x) {
        case Maybe::Something:
            std::cout << "it's something\n";
            break;
        case Maybe::Nothing:
            std::cout << "it's nothing\n";
            break;
        default:
            std::cout << "it's weird\n";
            break;
    }
    \end{lstlisting}
\end{frame}

\begin{frame}
    \frametitle{Control flow - iteration}
    \framesubtitle{Składniki języka C++}

    \begin{labeling}{{\tt do-while}}
        \item[{\tt while}] pętla wykonująca się dopóki wyrażenie kontrolne jest
            prawdziwe, ze sprawdzeniem przed wykonaniem instrukcji
        \item[{\tt do-while}] pętla wykonująca się dopóki wyrażenie kontrolne
            jest prawdziwe, ze sprawdzeniem po wykonaniu instrukcji
        \item[{\tt for}] pętla po zakresie
    \end{labeling}

    \vspace{1em}

    Pętla {\tt while} jest najbardziej ogólną pętlą, ale wszystkie rodzaje są
    równoważne (każdą z pętli da się zaimplementować w ramach każdej innej).
\end{frame}

\begin{frame}[fragile]
    \frametitle{Control flow - iteration ({\tt while})}
    \framesubtitle{Składniki języka C++}

    \begin{lstlisting}
    while (its_sunny_outside()) {
        std::cout << "weather is nice\n";
    }
    \end{lstlisting}
\end{frame}

\begin{frame}[fragile]
    \frametitle{Control flow - iteration ({\tt do-while})}
    \framesubtitle{Składniki języka C++}

    \begin{lstlisting}
    auto x = 0;
    do {
        x = roll_dice();
        std::cout << "you rolled " << x << "\n";
    } while (x != 6);
    \end{lstlisting}
\end{frame}

\begin{frame}[fragile]
    \frametitle{Control flow - iteration ({\tt for})}
    \framesubtitle{Składniki języka C++}

    \begin{lstlisting}
    for (auto i = 10; i >= 0; --i) {
        std::cout << i << '\n';
    }
    std::cout << "Happy New Year!\n";
    \end{lstlisting}
\end{frame}

\begin{frame}
    \frametitle{Control flow - procedural abstraction}
    \framesubtitle{Składniki języka C++}

    Funkcje spełniają rolę procedur w C++. Każda funkcja składa się z:

    \begin{enumerate}
        \item nazwy -- identyfikatora jakim można ją \emph{wywołać}
        \item listy parametrów formalnych -- specyfikacji jakich
            \emph{argumentów} (\emph{parametrów faktycznych}) wymaga od
            wywołującego
        \item typu zwracanego -- specyfikacji typu jakiego wartości funkcja
            produkuje
        \item ciała -- ograniczonego nawiasami klamrowymi zbioru instrukcji
            określającego operacje jakich dana funkcja jest abstrakcją
    \end{enumerate}
\end{frame}

\begin{frame}[fragile]
    \frametitle{Control flow - procedural abstraction}
    \framesubtitle{Składniki języka C++}

    \begin{lstlisting}
    auto add_one(int const x) -> int
    {
        return (x + 1);
    }
    \end{lstlisting}
\end{frame}

\begin{frame}
    \frametitle{Control flow - recursion}
    \framesubtitle{Składniki języka C++}

    Rekurencja jest realizowana za pomocą funkcji.
\end{frame}

\begin{frame}[fragile]
    \frametitle{Control flow - recursion}
    \framesubtitle{Składniki języka C++}

    \begin{lstlisting}
    /* Raises b to the power of n. */
    auto exponentiate(int const b, int const n) -> int
    {
        if (n <= 0) {
            return 1;
        }
        return (b * exponentiate(b, (n - 1)));
    }
    \end{lstlisting}
\end{frame}

\begin{frame}
    \frametitle{Control flow - concurrency and parallelism}
    \framesubtitle{Składniki języka C++}

    Współbieżność w C++ jest realizowana za pomocą \emph{wątków}. Tym samym
    mechanizmem jest realizowana \emph{równoległość} przetwarzania
    (\emph{parallelism}).

    \vspace{1em}

    Współbieżność można też zaimplementować na własną rękę, ale wymaga to
    znacznie większego nakładu pracy.
\end{frame}

\begin{frame}[fragile]
    \frametitle{Control flow - concurrency and parallelism ({\tt std::thread})}
    \framesubtitle{Składniki języka C++}

    {\footnotesize
    \begin{lstlisting}
    auto display_greeting(std::string const name) -> void
    {
        std::cout << ("Hello, " + name + "!\n");
    }

    auto t1 = std::thread{display_greeting, "Joe"};    // Armstrong
    auto t2 = std::thread{display_greeting, "Bjarne"}; // Stroustrup

    /*
     * Threads must be joined into the parent thread, or
     * the program will crash.
     */
    t1.join();  // joining thread is blocked until joined
                // thread terminates
    t2.join();
    \end{lstlisting}}
\end{frame}

\begin{frame}
    \frametitle{Control flow - nondeterminism}
    \framesubtitle{Składniki języka C++}

    Niedeterminizm jest nieodłączną cechą równoległości -- nie mamy gwarancji w
    jakiej kolejności będą względem siebie wykonywac sie operacje w
    \emph{różnych} wątkach.

    \vspace{1em}

    Niedeterminizm wewnątrz wątku możemy uzyskać generując liczby losowe. W tym
    celu można użyć {\tt std::ranom\_device} lub odczytać $n$ bajtów z pliku
    {\tt /dev/urandom}.
\end{frame}

\begin{frame}[fragile]
    \frametitle{Control flow - nondeterminism ({\tt std::random\_device})}
    \framesubtitle{Składniki języka C++}

    {\footnotesize
    \begin{lstlisting}
    std::random_device rd;
    std::uniform_int_distribution<int> d20 (1, 20);

    constexpr auto CRITICAL_SUCCESS = 20;
    constexpr auto CRITICAL_FAILURE = 1;

    auto const x = d20(rd);

    if (x == CRITICAL_SUCCESS) {
        std::cout << "you kill the monster in a single blow!\n";
    } else if (x == CRITICAL_FAILURE) {
        std::cout << "you wound yourself with your own sword!\n";
    } else {
        std::cout << "roll for damage.\n";
    }
    \end{lstlisting}}
\end{frame}

\begin{frame}
    \frametitle{Control flow - exceptions}
    \framesubtitle{Składniki języka C++}

    Mechanizmem dedykowanym sygnalizacji i obsługi błędów w C++ są
    \emph{wyjątki}. Wyjątek może być rzucony (zasygnalizowany) słowem kluczowym
    {\tt throw}; obsługa wyjątków odbywa się w bloku {\tt try-catch}.

    \vspace{1em}

    C++ pozwala na użycie dowolnego typu jako wyjątku.
\end{frame}

\begin{frame}[fragile]
    \frametitle{Control flow - exceptions}
    \framesubtitle{Składniki języka C++}

    {\footnotesize
    \begin{lstlisting}
    auto search_your_feelings(std::string father) -> void
    {
        if (father == "Darth Vader") {
            throw std::string{"NO!!! NO!!!"};
        }
    }

    /* ... */

    try {
        luke.search_your_feelings(lord_vader);
    } catch (std::string const& error) {
        std::cerr << ("operation failed: " + error + '\n');
    }
    \end{lstlisting}}
\end{frame}

\begin{frame}
    \frametitle{Data structures - biblioteka standardowa}
    \framesubtitle{Składniki języka C++}

    Biblioteka standardowa (ang. \emph{standard library}) języka C++ zawiera
    wiele struktur danych takich jak
    {\tt std::vector} (sekwencja o zmiennej długości),
    {\tt std::queue} (kolejka FIFO),
    {\tt std::map} (struktura mapująca klucze na wartości),
    {\tt std::string} (napis),
    {\tt std::pair}, itd.

    \vspace{1em}

    Warto używać struktur (a także funkcji) z biblioteki standardowej żeby
    oszczędzić sobie pracy.
\end{frame}

\begin{frame}
    \frametitle{Data structures - własne typy danych}
    \framesubtitle{Składniki języka C++}

    Programista C++ może definiować również własne typy danych: struktury i
    klasy, oraz wyliczenia.

    \vspace{1em}

    Klasy ({\tt class}) różnią się od struktur ({\tt struct}) tylko i wyłącznie
    tym, że ich pola są domyślnie publiczne.

    \vspace{1em}

    Wyliczenia słabe ({\tt enum}) są typu {\tt int}, ich wartości są globalne, i
    mają automatycznie zdefiniowane operacje arytmetyczne (np. sumę bitową). Są
    przydatne przy definiowaniu flag, które można łączyć.

    Wyliczenia silne ({\tt enum class}) różnią sie od słabych tym, że są
    ''swojego własnego'' typu, ich wartości nie są globalne, oraz nie mają
    automatycznie zdefiniowanych operacji arytmetycznych. Są przydatne przy
    definiowaniu rozdzielnych stanów.
\end{frame}

\begin{frame}[fragile]
    \frametitle{Data structures - struktury ({\tt struct})}
    \framesubtitle{Składniki języka C++}

    {\scriptsize
    \begin{lstlisting}
    struct being_with_legs {
        std::string const name;
        size_t const legs;

        being_with_legs(std::string, size_t);
    };

    being_with_legs::being_with_legs(std::string n, size_t l)
        : name{std::move(n)}
        , legs{l}
    {}

    /* ... */

    auto const snake = being_with_legs{ "snake", 0 };
    auto const human = being_with_legs{ "human", 2 };
    auto const spider = being_with_legs{ "spider", 8 };
    \end{lstlisting}}
\end{frame}

\begin{frame}[fragile]
    \frametitle{Data structures - wyliczenia ({\tt enum class})}
    \framesubtitle{Składniki języka C++}

    {\scriptsize
    \begin{lstlisting}
    enum class meal_kind {
        BREAKFAST,
        DINNER,
        SUPPER,
    };

    auto is_most_important_meal_of_the_day(meal_kind const meal) -> bool
    {
        return (meal == meal_kind::BREAKFAST);
    }
    \end{lstlisting}}
\end{frame}

\begin{frame}[fragile]
    \frametitle{Data structures - wyliczenia ({\tt enum})}
    \framesubtitle{Składniki języka C++}

    {\scriptsize
    \begin{lstlisting}
    enum some_flags_type {
        SOME_FLAG_READ,
        SOME_FLAG_WRITE,
        SOME_FLAG_NONBLOCK,
        SOME_FLAG_BUFFERED,
        SOME_FLAG_UNBUFFERED,
    };
    constexpr auto SOME_FLAG_DEFAULT = SOME_FLAG_READ
                                     | SOME_FLAG_WRITE
                                     | SOME_FLAG_BUFFERED;

    // We want read-only, non-blocking, unbuffered descriptor.
    auto const mode = SOME_FLAG_READ
                    | SOME_FLAG_NONBLOCK
                    | SOME_FLAG_UNBUFFERED;
    \end{lstlisting}}
\end{frame}

\begin{frame}
    \frametitle{I/O}
    \framesubtitle{Składniki języka C++}

    C++ korzysta z mechanizmów I/O dostarczanych przez API systemu operacyjnego
    (np. Linux), ale część z nich opakowuje w swoje własne abstrakcje
    zapewniając programom przenośność.
\end{frame}

\begin{frame}
    \frametitle{I/O - standard steams}
    \framesubtitle{Składniki języka C++}

    W momencie uruchomienia dla większości programów tworzone są 3 standardowe
    strumienie: wejścia, wyjścia, i błędów.

    \begin{labeling}{{\tt std::cerr}}
    \item[{\tt std::cin}] standardowy strumień wejścia, służy do odczytu danych
        podawanych przez użytkownika w konsoli tekstowej (\emph{file descriptor
        0})
    \item[{\tt std::cout}] standardowy strumień wyjścia, służy do prezentacji
        wyników działania programu w konsoli tekstowej (\emph{file descriptor
        1})
    \item[{\tt std::cerr}] standardowy strumień błędów, służy do prezentacji
        błędów działania i awarii programu w konsoli tekstowej (\emph{file
        descriptor 2})
    \end{labeling}
\end{frame}

\begin{frame}[fragile]
    \frametitle{I/O - standard steams}
    \framesubtitle{Składniki języka C++}

    {\scriptsize
    \begin{lstlisting}
    {
        std::string line;

        // read a line of text from standard input
        std::getline(std::cin, line);
    }

    // display a message to inform user of what is happening...
    std::cerr << "connecting to server...\n";

    // ...or notify them about errors
    std::cerr << "connection failed\n";

    // display results of program's work
    std::cout << downloaded_data << '\n';
    \end{lstlisting}}
\end{frame}

\begin{frame}
    \frametitle{I/O - files}
    \framesubtitle{Składniki języka C++}

    C++ definiuje typy {\tt std::ifstream} (\emph{input file stream}) i
    {\tt std::ofstream} (\emph{output file stream}) w bibliotece standardowej.

    \vspace{1em}

    Jeśli ich interfejs nie jest wystarczający zawsze można użyć interfejsu
    platformy, np. wywołań systemowych definiowanych przez standard POSIX --
    {\tt open(3)}, {\tt write(3)}, {\tt read(3)}, i {\tt close(3)}.
\end{frame}

\begin{frame}[fragile]
    \frametitle{I/O - files ({\tt std::ifstream} and {\tt std::ofstream})}
    \framesubtitle{Składniki języka C++}

    {\scriptsize
    \begin{lstlisting}
    auto path = std::string{"./data.txt"};

    {
        // write line to a file
        auto out = std::ofstream{ path };
        if (out.good()) {
            out << "Hello, World!\n";
        }
    }

    {
        // read line from a file
        auto in = std::ifstream{ path };
        if (in.good()) {
            auto line = std::string{};
            std::getline(in, line);
            std::cout << line << "\n";
        }
    }
    \end{lstlisting}}
\end{frame}

\begin{frame}
    \frametitle{I/O - sieć}
    \framesubtitle{Składniki języka C++}

    Komunikacja po sieci odbywa się z wykorzystaniem mechanizmów I/O
    dostarczanych przez standard POSIX --
    {\tt socket(2)},
    {\tt bind(2)},
    {\tt listen(2)},
    {\tt accept(2)},
    {\tt connect(2)},
    {\tt inet\_pton(3)},
    {\tt connect(3)},
    {\tt shutdown(3)},
    {\tt close(3)}.
\end{frame}

\begin{frame}[fragile]
    \frametitle{I/O - sieć (nagłówki)}
    \framesubtitle{Składniki języka C++}

    Pliki nagłówkowe, które należy dodać na początku pliku z kodem źródłowym
    żeby móc bez przeszkód korzystać z funkcji pozwalających na komunikację po
    sieci:

    {\small
    \begin{lstlisting}
    #include <arpa/inet.h>
    #include <endian.h>
    #include <netinet/ip.h>
    #include <string.h>
    #include <sys/types.h>
    #include <sys/socket.h>
    #include <sys/un.h>
    #include <unistd.h>
    \end{lstlisting}}
\end{frame}

\begin{frame}[fragile]
    \frametitle{I/O - sieć (klient)}
    \framesubtitle{Składniki języka C++}

    {\scriptsize
    \begin{lstlisting}
    auto sock = socket(AF_INET, SOCK_STREAM, 0);

    auto const ip = std::string{"127.0.0.1"};
    auto const port = uint16_t{42420};

    sockaddr_in server;
    memset(&server, 0, sizeof(server));
    server.sin_family = AF_INET;
    server.sin_port = htobe16(port);
    inet_pton(AF_INET, ip.c_str(), reinterpret_cast<void*>(&server.sin_addr));

    connect(sock, reinterpret_cast<sockaddr*>(&server), sizeof(server));

    auto const data = std::string{"Hello, World!"};
    write(sock, data.c_str(), data.size());

    shutdown(sock, SHUT_RDWR);
    close(sock);
    \end{lstlisting}}
\end{frame}

\begin{frame}[fragile]
    \frametitle{I/O - sieć (serwer, część $\frac{1}{2}$)}
    \framesubtitle{Składniki języka C++}

    {\tiny
    \begin{lstlisting}
    auto sock = socket(AF_INET, SOCK_STREAM, 0);

    auto const ip = std::string{"127.0.0.1"};
    auto const port = uint16_t{42420};

    sockaddr_in server;
    memset(&server, 0, sizeof(server));

    server.sin_family = AF_INET;
    server.sin_port = htobe16(port);

    inet_pton(AF_INET, ip.c_str(), reinterpret_cast<void*>(&server.sin_addr));

    bind(sock, reinterpret_cast<sockaddr*>(&server), sizeof(server));
    listen(sock, 0);  // default limit of waiting connections
    \end{lstlisting}}

    W ten sposób przygotuje się socket do odbierania połączeń.
\end{frame}

\begin{frame}[fragile]
    \frametitle{I/O - sieć (serwer, część $\frac{2}{2}$)}
    \framesubtitle{Składniki języka C++}

    {\tiny
    \begin{lstlisting}
    sockaddr_in client_addr;
    memset(&client_addr, 0, sizeof(client_addr));
    auto client_len = socklen_t{sizeof(client_addr)};
    auto client = accept(sock,
        reinterpret_cast<sockaddr*>(&client_addr), &client_len);
    {
        // report client's address
        std::array<char, INET_ADDRSTRLEN + 1> buffer {};
        inet_ntop(AF_INET, &client_addr.sin_addr, buffer.data(), buffer.size());
        std::cout << buffer.data() << ':' << be16toh(client_addr.sin_port) << "\n";
    }
    {
        // read data from client
        std::array<char, 512> buffer {};
        auto const n = read(client, buffer.data(), buffer.size());
        std::cout << std::string{buffer.data(), buffer.data() + n} << "\n";
    }

    shutdown(sock, SHUT_RDWR);
    close(sock);
    \end{lstlisting}}

    W ten sposób odbiera się połączenia i rejestruje adres połączonego klienta.
\end{frame}

\end{document}
